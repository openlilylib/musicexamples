%%%%%%%%%%%%%%%%%%%%%%%%%%%%%%%%%%%%%%%%%%%%%%%%%%%%%%%%%%%%%%%%%%%%%%%%%%%
%                                                                         %
%      This file is part of the 'openLilyLib' library.                    %
%                                ===========                              %
%                                                                         %
%              https://github.com/lilyglyphs/openLilyLib                  %
%                                                                         %
%  Copyright 2012-13 by Urs Liska, lilyglyphs@ursliska.de                 %
%                                                                         %
%  'openLilyLib' is free software: you can redistribute it and/or modify  %
%  it under the terms of the GNU General Public License as published by   %
%  the Free Software Foundation, either version 3 of the License, or      %
%  (at your option) any later version.                                    %
%                                                                         %
%  This program is distributed in the hope that it will be useful,        %
%  but WITHOUT ANY WARRANTY; without even the implied warranty of         %
%  MERCHANTABILITY or FITNESS FOR A PARTICULAR PURPOSE. See the           %
%  GNU General Public License for more details.                           %
%                                                                         %
%  You should have received a copy of the GNU General Public License      %
%  along with this program.  If not, see <http://www.gnu.org/licenses/>.  %
%                                                                         %
%%%%%%%%%%%%%%%%%%%%%%%%%%%%%%%%%%%%%%%%%%%%%%%%%%%%%%%%%%%%%%%%%%%%%%%%%%%

% Contributor's Guide - Tutorial part
%
%%%%%%%%%%%%%%%%%%%%%%%%%%%%%%%%%%%%%%%%%%%%%%%%%%%%%%%%%%%%%%%%%%%%%%%%%%%

\documentclass[openLilyLib_CG]{subfiles}

\begin{document}

\chapter{Directory structure, Document Class(es) etc.}

\begin{authorAbstract}{Urs Liska}
To be done.\\
Describe the main books, and how the parts are included into them using the \package{subfile} approach.\\
Describe the documentclass(es), list the included packages
\end{authorAbstract}




\chapter{Command Reference}
\begin{authorAbstract}{Urs Liska}
This is the reference on specific commands that have been implemented for \openlilylib's \LaTeX{} documentation.
It is somewhat sketchy, but I want all commands at least to be enumerated as soon as they are implemented.
\end{authorAbstract}

\section{Formatting}
One should \emph{never} apply manual formattings but always semantic markup through the use of commands.
If there is no suitable command available then an author should define one himself or ask for new ones.

\subsection{Environments}
\begin{description}
\item[\env{authorAbstract}] Although \openlilylib{} is a collaborative effort, we want to attribute the original author of tutorials or noteable chunks of documentation.
This environment should be used after a sectioning command.
The original author is given as the mandatory argument to the environment.
Contributors may be given as an optional argument (use a comma-separated list of \cmd{contributor} items).
The contents of the environment is formatted as an abstract.
\todo{It is intended to implement an index of original authors and their contributions.}\\
If you want to mark a substantial contribution in a place where an abstract seems inappropriate use the command \cmd{originalAuthor} instead.
\item[\env{knownIssues}] Use this environment to print a section with -- well -- known issues and warnings \dots
\end{description}

\subsection{Paragraph Formatting}
\begin{description}
\item[None yet]
\end{description}

\subsection{Character Formatting}

\begin{description}
\item[\cmd{cmd}] Commands that start with a backslash can be printed using this command. It switches the font and prints the backslash in front of the command name.
\item[\cmd{env}] Environment names are highlighted using this command.
\item[\cmd{package}] You can enter package names with this command.
\item[\cmd{todo}] Any TODO item should be entered in the source using this command, because it will guarantee a consistent appearance and it's quite spottable.
This may someday be enhanced to create an index.
\end{description}

\section{Other Commands}
\begin{description}
\item[\cmd{contributor}] Use this command together with \cmd{originalAuthor} or the \env{authorAbstract} environment.
So far this is only a formatting command, but it is intended to create an index of contributors and their contributions (see \ghIssue{7})
\item[\cmd{ghIssue}] Use this command to create a link to an issue report on \openlilylib's Github site.
Pass the plain issue number (without leading zeroes or other entities) as the mandatory argument.
\item[\cmd{openlilylib}] Use this command to print \openlilylib's name.
This may be formatted more stylishly later \dots
\item[\cmd{originalAuthor}] This is a stripped down command version of the \env{authorAbstract} environment.
You can use this command to indicate authorship for some piece of text without the need for providing an abstract.
The mandatory argument is the author's name, and you may supply a comma-separated list of \cmd{contributor} entries as the optional argument.
\end{description}

\section{Music and Code Examples}
Inserting music examples can be achieved using \package{musicexamples}.
This package is part of \openlilylib{} and is included by the document class.
Please refer to the package documentation for how to work with it.

\todo{Code listings (LilyPond or \LaTeX) haven't been implemented yet}

\end{document}
