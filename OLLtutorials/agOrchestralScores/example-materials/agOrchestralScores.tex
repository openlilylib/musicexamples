% Use this or any other documentclass you like
\documentclass{scrreprt}

\usepackage{verbatim}

% include OLL's base files (in their present state)
\usepackage{OLLbase}
\usepackage{OLLstyles}

% Style file for command you define along writing
\usepackage{agStyles}

\begin{document}
\chapter{Orchestral Scores with LilyPond}
\originalAuthor{Antonio Gervasoni}

\begin{abstract}
About a year ago, I became interested in writing an orchestral score using Lilypond and wanted a tutorial that could guide me through the process step by step. 
Being unable to find one on the Internet, I embarked myself in the task of putting together ideas from blogs, forums, and from the Lilypond documentation, related to the matter.
I went back and forth, trying different approaches and finally, I found my way through the maze of possibilities and the score of my work took shape.
As a result of this, I developed a procedure that could be followed by other users facing the same issue.
If you want to write an orchestral score using Lilypond, I'm sure you will find this tutorial useful.
\end{abstract}

\section{Introduction}

If you are reading this document then you must be someone with an interest in typesetting orchestral scores using Lilypond.
You may be an experienced Lilypond user who, until now, has only created simple scores for just a few instruments and wants to know how to deal with the typesetting of a whole orchestral score, or maybe one who has already done this several times and is just curious about how other users tackle this kind of job.
On the other hand, you may be a beginner; someone who has discovered Lilypond very recently and wants to start right away with using it to compose or transcribe an orchestral piece.

A word of notice: if you have no experience with Lilypond at all, I would advise you to at least read the Learning Manual - which is available at Lilypond's official website (\url{www.lilypond.org}) - before you continue reading this tutorial.

I discovered Lilypond on december of 2011 and immediately used it in a small composition I wrote at the time. Then, in march 2012, I decided to put Lilypond to the test and also to study it's code in further deep.
The best way to do this, I thought, would be to typeset an old piece I composed in 2003 named Icarus.

I was already thinking of making a revision of the piece and, at the same time, the work presented itself as a real challenge because of it's complexity. 
It's an atonal piece, very chromatic, with constant time signature changes and intricate counterpoint.

Lilypond is incredibly flexible, and while this may be one of the reasons it is so powerful, it also means it can be quite confusing for the user, especially for the beginner.
It's Learning Manual and Notation Reference are both quite notable documents, full of examples and detailed explanations, but a more specialized tutorial is needed in order to write a work for orchestra.

I searched websites and blogs, looking for ideas on how to start such a job and, while I did find various posts that finally led me in the right direction, I was unsuccessful in finding a document like the one you are reading right now, where a Lilypond user presents a full description of how to write an orchestral score using Lilypond.

In the following pages I will show you how  to proceed in order to write a score for orchestra, using my composition as an example.
However, I need to point out that I'm not an expert in Lilypond.
I'm fairly new to it, with just one year of experience, compared to other users who may have several years using it.
Therefore, don't expect this method to be perfect or compact.
In fact, I may have been doing some stuff the hard way, only because I don't know an easier way yet.
Hopefully, other users will find ways to improve this method so this tutorial may gradually evolve into a more refined procedure to write orchestral scores.

Before I start, I'd like to thank all those users who, by sharing their ideas on the Internet, have allowed me to typeset my piece and to write this tutorial.
I apologize for not taking the care to write down all their names (I do remember some!).
Therefore, if you read this document and are able to identify something which you can positively identify as being from your own authorship (or from others as well), please write me an e-mail to agervasoni@gmail.com and I'd be happy to cite your name (or names, if there is more than one author) where appropriate.

One thing you might want to do before you keep on reading is to download the sheet music and the .ly files of my work, Icarus.

I have uploaded the score and parts as pdf files to IMSLP (Internet Music Score Library Project):

The Mutopia Project has both the pdf and the .ly files as well:

You can also hear the computer sequence at my SoundCloud channel:

\section{The First Steps}

So, you want to write a piece for orchestra using Lilypond and you don't know where to start.
Using my work Icarus as an example, we will begin with the first instrument (part) in the score, the one on top of all the others: the flute.
Of course, a full mid-size orchestra has two flutes and they are usually written in the same stave but for now we'll just focus on the first flute and leave the other flute behind for the moment.
We will see later how to include it.

Let's begin by creating a file with all the blocks we need to create a score for the first flute. Here is the code:

\begin{lilypondcode}
\version 2.16.1 %or whatever other version of Lilypond you are using

\header {
  title = "Icarus" 
  composer = "Antonio Gervasoni"
}

\language "nederlands"

\include "IcarusMetrics.ly"
\include "../Definitions/MyDefinitions.ly"

InstrumentHeading = {
  #(set-accidental-style 'modern)
  \override Staff.Accidental #'hide-tied-accidental-after-break = ##t
  \set Staff.instrumentName = \markup \bold { "Instrument Name " }
}

InstrumentNotes = {
  \numericTimeSignature
  \clef treble
  \relative c' {
    The music goes here
  }
}

\score {
  <<
    \new Staff = "instrument" <<
      %\compressFullBarRests
      \Metrics 
      \InstrumentHeading
      \InstrumentNotes %% or \transpose x y { \InstrumentsNotes }
    >>
  >>
  \layout { }
}

\paper {
}
\end{lilypondcode}

The first thing you may have noticed are two obscure include commands. 
The first one includes a file called IcarusMetrics.ly. 
As I already explained, my piece has constant time signature changes.
Having to write them all again in every score would be tedious and will ultimately cost me precious time.
Therefore, I created a file with all those changes to be included in every score, so I don't have to write them down repeatedly.
The first lines of this file look like this:

\begin{lilypondcode}
\version "2.16.1"

Metrics = {
  \time 4/4 \tempo 4 = 55 \skip 1*8 | % 1-8
  \time 2/4 \skip 2 | % 9 \mark \default %{ A %}
  \skip 2 | % 10
  \time 3/4 \skip 2. | % 11
  \time 2/4 \skip 2 | % 12
  .
  .
  .
  etc.
}
\end{lilypondcode}

As you can see, it is just a series of time signature changes and tempo markings separated by skips. 
The complete file can be found amongst the files uploaded to the Mutopia Project.

The second file which I have included is called MyDefinitions.ly and is located one tree level above the present location of the score, in a folder named Definitions, which is the reason why the name of the file is preceded by \verb|"../Definitions"|.
This file contains the code of all the tweaks that I use throughout my scores. 
The first lines of this file look like this:

\begin{lilypondcode}
\version "2.16.1"

%%%%%%%%%%%%%%%%%%%%%%%%%%%%%%%%%%%%%%%
%% Tuplets
bracketUp = \override TupletBracket #'direction = #UP
bracketNeutral = \revert TupletBracket #'direction
bracketDown = \override TupletBracket #'direction = #DOWN
hideTupletBracket = \once \override TupletBracket #'stencil = ##f 
hideTupletNumber = \once \override TupletNumber #'stencil = ##f

%%%%%%%%%%%%%%%%%%%%%%%%%%%%%%%%%%%%%%%
%% Rests
restMiddle = \override Rest #'staff-position = #0
restNeutral = \revert Rest #'staff-position
multirestMiddle = \override MultiMeasureRest #'staff-position = #-0.01
multirestFourthLine = \override MultiMeasureRest #'staff-position = #2
multirestNeutral = \revert MultiMeasureRest #'staff-position
\end{lilypondcode}

The whole point about creating definitions - known in Lilypond as "variables" - is that they make it easier for us to perform a particular task.
Instead of writing a whole bunch of code every time we need to do something, we just have to enclose that code inside a variable, and write it's name where we need the task to be performed.

As we go through each part in the score, I will make notice of any special definition I have used for it's creation and at the end of this tutorial I'll include a complete list with all of them.

\pagebreak

\section{Referenced section}
\label{sec:ref_sec}

\end{document}