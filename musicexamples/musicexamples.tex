%%%%%%%%%%%%%%%%%%%%%%%%%%%%%%%%%%%%%%%%%%%%%%%%%%%%%%%%%%%%%%%%%%%%%%%%%%%
%                                                                         %
%      This file is part of the 'openLilyLib' library.                    %
%                                ===========                              %
%                                                                         %
%              https://github.com/lilyglyphs/openLilyLib                  %
%                                                                         %
%  Copyright 2012-13 by Urs Liska, lilyglyphs@ursliska.de                 %
%                                                                         %
%  'openLilyLib' is free software: you can redistribute it and/or modify  %
%  it under the terms of the GNU General Public License as published by   %
%  the Free Software Foundation, either version 3 of the License, or      %
%  (at your option) any later version.                                    %
%                                                                         %
%  This program is distributed in the hope that it will be useful,        %
%  but WITHOUT ANY WARRANTY; without even the implied warranty of         %
%  MERCHANTABILITY or FITNESS FOR A PARTICULAR PURPOSE. See the           %
%  GNU General Public License for more details.                           %
%                                                                         %
%  You should have received a copy of the GNU General Public License      %
%  along with this program.  If not, see <http://www.gnu.org/licenses/>.  %
%                                                                         %
%%%%%%%%%%%%%%%%%%%%%%%%%%%%%%%%%%%%%%%%%%%%%%%%%%%%%%%%%%%%%%%%%%%%%%%%%%%

\documentclass{OLLbook}

\title{musicexamples \LaTeX{} package}
\author{Urs Liska}

\begin{document}
\maketitle
\begin{authorAbstract}{Urs Liska}
\texttt{musicexamples} is a set of tools intended for printing and managing music examples in \LaTeX{} documents.
It was developed with examples in mind that are produced using the LilyPond notation software%
\footnote{\url{www.lilypond.org}},
but it can also be used to handle any kind of images.

It consists of three parts: a \LaTeX{} package, a set of configuration files for LilyPond scores and a set of Python scripts (to be implemented).
\end{authorAbstract}

\vfill
\input{copyright-notice.inp}

\tableofcontents

\chapter{End User Documentation}
\section{Installation and Requirements}

\section{musicexamples.sty}

\texttt{musicexamples} is a package that defines environments and commands to handle music examples (scores and fragments) within \LaTeX{} documents.
It supports floating or non-floating examples, one- or multi-system examples and finally full-(one- or multi-)page scores to be inserted.
The examples are numbered in one list and can be output as one contigious list of music examples, regardless of their type of inclusion.

It was developed from the perspective of a user of the LilyPond notation software, but the package should work with any kind of image suitable for music examples.
The package has some parallels with LilyPond's \texttt{lilypond-book} scripts, but it doesn't understand itself as a competitor for this, but rather as a different approach for people with somewhat different needs.
For anybody writing (about) music it may also be a good idea to have a look at my \texttt{lilyglyphs} package that will eventually be merged into the openLilyLib family of resources.

\bigskip
In order to use \texttt{musicexamples} you simply write \cmd{usepackage\{musicexamples\}} or \cmd{RequirePackage\{musicexamples\}}.
You will then have access to the its commands and environments:

There are two environments to be used for music examples within a page: \texttt{musicexample} and \texttt{musicexampleNonFloat}.
The point for having a non-floating environment is \emph{not} to have more control over the placement of the item, but rather to allow it to cross page breaks, so that a group of music systems may flow over one or more pages. 
These environments do not print the music examples themselves but only provide the environment for them (as a \texttt{figure} environment doesn't already print the figure).
You use them like any other float environment, so you can optionally add the placement directive after the \cmd{begin} statement.
Inside the environment you add the contents (see below) and the \cmd{caption} to be used.
\begin{quote}
\begin{verbatim}
\begin{musicexample}[t]
  \includegraphics{exampleimage}
  \caption{A typical music example}
\end{musicexample}
\end{verbatim}
\end{quote}

This will print your image in a floating environment [preferrably at the top of a page], will take care of the numbering and prepares for the inclusion in a list of music examples.

The usage of \texttt{musicexampleNonFloat} is identical, except that it doesn't accept the optional placement argument.
It will print the example right where you inserted the environment, and while the floating version can only print the example on one page, this one can spread over page breaks -- provided the music systems are given as a series of images.

The captions are (to my knowledge) standard captions that can be influenced (formatted) with the commands of the \texttt{caption} package, the default layout and style being due to my needs when developing the package.
The caption label defaults to “Music Example”, the heading of the list to “List of Music Examples”.
In order to change them you can use the commands \cmd{setXmpCaptionLabel} and \cmd{setXmpListName} with one mandatory argument supplying the respective string.
\begin{quote}
\begin{verbatim}

\end{verbatim}
\end{quote}


\begin{itemize}
\item \texttt{musicexample}
\item \texttt{musicexampleNonFloat}
\item \cmd{onePageMusicExample}
\item \cmd{onePageLilyPondExample}
\item \cmd{multiPageMusicExample}
\item \cmd{multiPageLilyPondExample}
\end{itemize}

\section{The LilyPond Configuration Files}

\section{The Python Scripts}

\chapter{Implementation}

\section{musicexamples.sty}

\section{LilyPond Files}

\chapter{Licenses}
\input{licenses.inp}

\end{document}